\section{Java пакети}

\begin{frame}\frametitle{Import Изрази}

\begin{itemize}
\item
  Со \texttt{import} изразите се вклучуваат предефинирани пакети и класи од
  Java, како што се \texttt{Scanner} од пакетот \texttt{java.util} \\
  \texttt{java.util.Scanner;}
\item
  Можно е да се вклучат сите класи од одреден пакет со: \texttt{import java.util.*;}
\item
  Притоа не постои дополнително оптоварување
\end{itemize}
\end{frame}

\begin{frame}\frametitle{Изразот package}

\begin{itemize}
\item
  За да се направи пакет, потребно е да се групираат сите класи заедно во еден
  единствен директориум (folder) и да се додаде следната наредба на почетокот
  на секоја класна датотека: \texttt{package package\_name};
\item
  Едниствено \texttt{.class} датотеките мора да бидат во тој директориум, додека
  \texttt{.java} датотеките се опционални
\item
  Единствено празни линии и коментари може да им претходат на изразот \texttt{package}
\item
  Ако постојат и \texttt{import} и \texttt{package} изрази, \texttt{package}
  изразите мора да им претходат на \texttt{import} изразите
\end{itemize}
\end{frame}

\begin{frame}\frametitle{Пакетот \texttt{java.lang}}

\begin{itemize}
\item
  Пакетот \texttt{java.lang} ги содржи основните класи на програмскиот јазик Java
  \item
    Се вклучува автоматски
  \item
   Некои од класите кои се наоѓаат во овој пакетот \texttt{java.lang} се:
   \texttt{Math}, \texttt{String} и останати \texttt{wrapper} класи
  \end{itemize}
\end{frame}

\begin{frame}\frametitle{Имиња на пакети и директориуми}

\begin{itemize}
\item
  Име на пакетот е името на патеката на директориумот или подиректориумите кои
  ги содржат класите
\item
  На Java и се потребни две нешта за да го најде директориумот за одреден пакет:
  името на пакетот и вредност на променливата \texttt{CLASSPATH}
  \begin{itemize}
  \item
    Системската променлива \texttt{CLASSPATH} е слична на променливата
    \texttt{PATH} и се поставува на истиот начин во оперативниот систем
  \item
    Променливата \texttt{CLASSPATH} е се поставува на листата на директориуми
    (вклучувајќи го и тековниот директориум, ``.'') во кој Java ќе бара пакети
    на одреден компјутер
  \item
    Java ги пребарува овие директориуми редоследни и го користи првиот
    директориум во кој ќе го најде одреден пакет
  \end{itemize}
\end{itemize}
\end{frame}

\begin{frame}\frametitle{Пакетот \texttt{default}}

\begin{itemize}
\item
  Сите класи во тековниот директориум спаѓаат во неименуван пакет наречен \texttt{default}
\item
  Се додека тековниот директориум (.) е дел од променливата \texttt{CLASSPATH},
  сите класи во овој пакет автоматски се достапни во програмата
\end{itemize}
\end{frame}

\begin{frame}[fragile]\frametitle{Колизии на имиња}

\begin{itemize}
\item
  Со цел да се чуваат организирани библиотеките со класи, пакетите овозможуваат
  соодветно справување со колизиите во имињата (ситуација кога две класи имаат
  исто име)
  \begin{itemize}
  \item
    Различни програмери кои пишуваат различни пакети може да искористат
    исто име за една или повеќе нивни класи
  \item
    Ова се разрешува со користење на целосното име на класата (составено од
    името на пакетот + името на класата) со цел да се разликуваат
\begin{verbatim}
package_name.ClassName
\end{verbatim}
  \item
    Ако се употребува целосното име, тогаш не е потребно да се вклучува
    пакетот
  \end{itemize}
\end{itemize}

\end{frame}

\section{javadoc}

\begin{frame}{Што е \texttt{javadoc}?}

\begin{itemize}
\item
  За разлика од јазики како C++, во Java интерфејсот и имплементацијата на
  класата се наоѓа во иста датотека
\item
  Меѓутоа, во Java постои програма наречена \texttt{javadoc} која автоматски
  го извлекува интерфејсот од класта и создава документација
  \begin{itemize}
  \item
    Овие информации се презентираат во HTML формат
  \item
    Коректното коментирање на одредена класа овозможува програмерите да се
    повикуваат на документацијата на нејзиното API (Application Programming
    Interface) за полесно користење
  \item
    \texttt{javadoc} создава документација од една единствена класа, па се до
    цели пакет од класи
  \end{itemize}
\end{itemize}
\end{frame}

\begin{frame}\frametitle{Коментирање класи за javadoc}

\begin{itemize}
\item
  \texttt{javadoc} ги извлекува заглавјата на класите, на коментарите, на сите
  јавни методи, класни променливи и статички променливи
\item
  Во стандардниот мод не се извлекуваат тела на методи или приватни методи
\item
  За да се извлече коментар, мора да се запази следното правило:
  \begin{itemize}
  \item
    Коментарот мора да претходи на јавна класа или дефиниција на метод
  \item
    Коментарот мора да биде блок (block comment) и неговото отварање /* мора да
    биде со дополнително * ( /** . . . */ )
  \end{itemize}
\end{itemize}
\end{frame}

\begin{frame}[fragile]\frametitle{Коментирање класи за javadoc}

\begin{itemize}
\item
  Дополнително на основните информации, коментарот кој претходи на дефиниција
  на јавен метод треба да содржи опис на параметрите, вредноста која се враќа
  и да ги пријави сите потенцијални исклучоци
\item
  На овој вид на информации им претходи знакот @ и се наречени @tag или @tags
\end{itemize}
\begin{lstlisting}
/** 
	General Comments about the method . . . 
    @param aParameter Description of aParameter 
    @return What is returned . . .
*/
\end{lstlisting}  
\end{frame}

\begin{frame}[fragile]\frametitle{Извршување на \texttt{javadoc}}

\begin{itemize}
\item
  Извршувањето на \texttt{javadoc} на одреден пакет се прави со следната
  команда: 
  \begin{verbatim}
  javadoc --d Documentation_Directory Package_Name
  \end{verbatim}
  \begin{itemize}
  \item
    HTML документите ќе се генерираат во директориумот \texttt{Documentation\_Directory}
  \item
    Ако \texttt{--d} и \texttt{Documentation\_Directory} се испуштат,
    \texttt{javadoc} ќе создаде соодветна структура на директориуми
  \end{itemize}
\item
  Извршувањето на \texttt{javadoc} на една класа се прави со командата:
  \begin{verbatim}
  javadoc ClassName.java
  \end{verbatim}
\item
  Ако сакаме да ја извршиме на сите класи во тој директориум:
 \begin{verbatim}
  javadoc *.java
  \end{verbatim}
\end{itemize}

\end{frame}
